\section{Problem 2}
\textbf{For a $\mathrm{p}^{+}-\mathrm{n}-\mathrm{p}$ bipolar junction transistor (BJT) with doping of the emitter, base and collector regions such that $N_E>N_B>N_C$.}
\subsection*{a) Sketch the band diagram for this device (qualitative accuracy is sufficient).}

\importimagewcaptionw{band}{Band diagram}{1}

\subsection*{b) Sketch the minority carrier distribution in the emitter, base and collector when the BJT operates in the normal active mode and indicate the space charge regions (qualitative accuracy is sufficient).}

\importimagewcaptionw{distribution}{Minority carrier distribution}{1}

\subsection*{c) In the following, assume that the base current is mainly due to recombination in the base and that $W_b<<L_p$ and $\Delta p_E>>p_n$. Use the charge control approach to find an expression for the base current (explain your reasoning).}
The injected hole current at $x_n = 0$ needed to maintain
the distribution is simply the total charge divided by the average time of
replacement:

$$
I_p\left(x_n=0\right)=\frac{Q_p}{\tau_p}=q A \frac{L_p}{\tau_p} \Delta p_n=q A \frac{D_p}{L_p} \Delta p_n
$$

As the base current is mainly due to recombination in the base this gives us

\subsection*{d) The exact solution for the base current is}

\begin{equation}
    I_B=q A \frac{D_p}{L_p}\left[\left(\Delta p_E+\Delta p_C\right) \tanh \frac{W_B}{2 L_p}\right]
    \label{eq:i}
\end{equation}


\textbf{For the same assumptions, show that equation (\ref*{eq:i}) is to a first order approximation equal to the expression from c).}