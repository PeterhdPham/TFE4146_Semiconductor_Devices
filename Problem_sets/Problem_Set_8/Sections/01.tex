\section{Problem 1}

\textbf{Figure \ref{fig:capacitance_vs_voltage.png} shows the high frequency capacitance vs. voltage ( $C-V$ characteristics) of an n-channel MOSFET, made from p-type Si substrate with a dopant concentration of $N_a=5 \cdot 10^{15} \mathrm{~cm}^{-3}$ and use of $\mathrm{n}^{+}$-polysilicon gate electrode. The thickness of the oxide in the gate area is $d_{o x}=100\text{Å}$, and the effective interface charge $Q_i=4 \cdot 10^{10} q\left[\mathrm{C} \mathrm{cm}^{-2}\right]$. Assume room temperature: $T=295$ $\mathrm{K}$.}

\importimagewcaption{capacitance_vs_voltage.png}{High frequency capacitance vs. voltage of an n-channel MOSFET}

\subsection*{a) A MOSFET of this kind has two contributions to the capacitance: $C_i$ and $C_s$. Describe the mechanisms behind these two contributions, draw an equivalent circuit diagram for the MOS capacitance and find an expression for $C_{\min }$ expressed by $C_i$ and $C_s=C_{d, \min }$ (assuming that the component at hand is biased in strong inversion).}

We start by describing the to capacitance:
\begin{itemize}
    \item The insulator capacitance $C_i$:
    \subitem This capacitance is given by $C_i=\frac{\epsilon_i}{d}$ where $\epsilon_i$ is the permittivity of the insulator, while $d$ is the insulator thickness. This capacitance is associated woth the metal-oxide part of the mos structure. It behaves a lot like a parallel-plate capacitor as the oxide layer functions as the dielectric, and the capacitance is determined by the permittivity of the oxide material and the thickness of the oxide layer.
    \item The voltage-dependent semiconductor capacitance $C_s$:
    \subitem This capacitance is given by $C_s=\frac{qQ}{qV}=\frac{dQ_s}{d\phi_s}$ where $Q_s$ is the space charge density per unit area and $\phi_s$ is the surface potential. This capacitance arise from the formation of a space-charge region withing the semiconductor when a voltage is applied. For less negative voltage the accumulation of electrons at the surface makes the semiconductor surface becomes depleted and a depletion-layer capacitance us added in series with $C_i$
\end{itemize}

\importimagewcaption{NMOS_transistor.png}{Circuit diagram for the MOS capacitance}
The minimum MOS capacitance $C_{min}$ is given by 

$$C_{min}= \frac{C_iC_{d,min}}{C_i+C_{d,min}}$$

as $C_s=C_{d,min}$ we get

$$C_{min}= \frac{C_iC_{s}}{C_i+C_{s}}$$

\subsection*{b) Determine a numerical value for the capacitance of the MOS structure at hand operating at a negative gate voltage $\left(V_G \ll V_T\right)$ and positive gate voltage $\left(V_G \gg V_T\right)$, respectively.}

For the case  $\left(V_G \ll V_T\right)$:

$$C=C_i=\frac{\epsilon_i}{d}$$

as the relative permittivity $\epsilon_r$ for silicon dioxide is 3.9 and permittivity of free space $\epsilon_0=8.85\cdot10^{-14}\frac{F}{cm}$ this gives us:

$$C=\frac{3.9\cdot8.85\cdot10^{-14}}{10^{-6}}=3.4515\cdot 10^{-7}$$

For the case  $\left(V_G \gg V_T\right)$:

$$C=C_{min}= \frac{C_iC_{s}}{C_i+C_{s}}$$

where $C_s=\frac{\epsilon_s}{W_m}$. We start by calculating $W_m$


$$W_m=2\left[\frac{ \epsilon_s \phi_F}{q N_a}\right]^{1 / 2}$$

where $\phi_F$ is given by

$$\phi_F=\frac{kT}{q}\ln \frac{N_a}{n_i}= \frac{1.38\cdot10^{-23}\cdot295}{1.6\cdot10^{-19}}\ln \frac{5\cdot10^{15}}{1.5\cdot10^{10}}=0.324eV$$

this gives

$$W_m=2\left[\frac{11.68\cdot8.85\cdot10^{-14} \cdot 0.324}{1.6\cdot10^{-19} \cdot5\cdot10^{15}}\right]^{1 / 2}=4.1\cdot10^{-5}cm$$

$$\Rightarrow C_s=\frac{11.68\cdot8.85\cdot10^{-14}}{4.1\cdot10^{-5}}=2.52\cdot10^{-8}$$

and we end up with

$$C= \frac{3.4515\cdot 10^{-7}\cdot2.52\cdot10^{-8}}{3.4515\cdot 10^{-7}+2.52\cdot10^{-8}}=2.35\cdot10^{-8}\frac{F}{cm^2}$$

\subsection*{c) Draw a sketch that shows how the $C-V$ characteristics in the figure above changes for low frequencies (typical $\sim 1-100 \mathrm{~Hz}$ ), and explain the reason for this characteristic change in the $C-V$ characteristics.}
Given:
\begin{itemize}
    \item Intrinsic charge carrier density for $\mathrm{Si}: n_i=1.5 \cdot 10^{10} \mathrm{~cm}^{-3}$
    \item Relative dielectric permittivity for $\mathrm{Si}: \varepsilon_r=11.8$
    \item Relative dielectric permittivity for $\mathrm{SiO}_2: \varepsilon_r=3.9$
\end{itemize}
\importimagewcaptionw{1c.jpg}{Sketch of the C - V characteristic.}{1}

