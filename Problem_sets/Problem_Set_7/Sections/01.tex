\section{Problem 1}
\textbf{The component in figure 1 consists of an n-type semiconductor with uniform doping and ohmic contacts at the ends (top/bottom) and sharp (abrupt) $p^+-n$ junctions with ohmic contacts at the edges. The length of the $p^+$ regions is L and the  istance between these region is marked by d.}
\subsection*{a) Assume that the terminals E-C and B-D are connected, and that $\mathrm{d}>>L_p{ }^1$.}

\subsubsection*{i) What standard semiconductor device does the figure represent under these conditions?}
This is a pn junction
\subsubsection*{ii) Sketch the current $\mathrm{I}_{E B}$ as a function of applied voltage $\mathrm{V}_{E B}$ between the connected terminal pairs E-C and B-D}
\importimage{1.jpg}
\subsection*{b) Assume now that only the terminals E-C are connected. Assume bias voltage $\mathrm{V}_{B D}>0$ and $\mathrm{V}_{E B}<0$.}
\subsubsection*{i) What standard semiconductor device does the figure represent under these conditions?}
JFET
\subsubsection*{ii) Sketch the current $\mathrm{I}_D$ as a function of the applied voltage $\mathrm{V}_{D B}$ for different levels of $\mathrm{V}_{E B}$}
\subsubsection*{iii) Indicate in the sketched I-V characteristics the effect of avalanche breakdown in the two $\mathrm{p}^{+}-\mathrm{n}$ junctions.}
\importimage{2.jpg}