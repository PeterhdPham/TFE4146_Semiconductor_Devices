\section{Problem 1 - Conductivity}
\subsection*{a)}
\textbf{The conductivity of a semiconductor will depend on the doping concentration. Express the conductivity $\sigma$ as a function of}

\begin{equation*}
     \tilde{n}=\frac{n_{0}}{n_{i}} 
\end{equation*}

\textbf{where $n_i$ is the intrinsic electron concentration, $n_0$ is the elecron concentration at thermal equilibrium.}

The general equation for conductivity due to electron is 

\begin{equation*}
    \sigma=q n_0 \mu_{n}
\end{equation*}

from the given function we have that 

\begin{equation*}
    n_0=\tilde{n}\cdot n_i
\end{equation*}

This gives us

\begin{equation*}
    \sigma=q (\tilde{n}\cdot n_i) \mu_{n} 
\end{equation*}

where $q$ is the charge of an electron, $\tilde{n}$ is the electron concentration normalized to the intrinsic value $n_i$ the intrinsic carrier concentration, $\mu_{n}$ is the electron mobility.

\textbf{Use the values of $\mu_n$ and $\mu_p$ for Si and plot the function for $\tilde{n} \in [0, 10]$. Give an intuitive explanation for the shape of the function. Do not consider the variations in the mobility with the doping concentration.}

\importimage{1a.png}

We can se that the relation between the conductivity $\sigma$ and the  electron concentration normalized to the intrinsic value is linear. It makes sense as the the conductivity increases when you n-dope the Si

\subsection*{b)}
\textbf{Show that the minimum conductivity is given by}

\begin{equation*}
    \sigma_{\min }=2 q n_{i} \sqrt{\mu_{n} \mu_{p}}
\end{equation*}

The current density can be written in terms of mobility as

\begin{equation*}
    J_{x}=q n \mu_{n} \mathscr{E}_{x} 
\end{equation*}

If both electrons and holes participate, then we must modify the equation to

\begin{equation*}
    J_{x}=q\left(n \mu_{n}+p \mu_{p}\right) \mathscr{E}_{x}=\sigma \mathscr{E}_{x}
\end{equation*}

by combining this with the product of electron and hole concentration and assuming that the conductivity is at minimum when $n_0=p_0=n_i$.

\begin{equation*}
    n_0 p_o=n^2_i
\end{equation*}

\begin{equation*}
    p_0 = \frac{n_i^2}{n_0}
\end{equation*}

Substitute this into $ \sigma $:

\begin{equation*}
    \sigma = q(n_0 \mu_n + \frac{n_i^2}{n_0} \mu_p)
\end{equation*}

for $\sigma$ to be minimized, $\frac{d\sigma}{dn_0}=0$

\begin{equation*}
    \frac{d\sigma}{dn_0} = q(\mu_n - \frac{n_i^2 \mu_p}{n_0^2})=0
\end{equation*}

\begin{equation*}
    \mu_n = \frac{n_i^2 \mu_p}{n_0^2}
\end{equation*}
\begin{equation*}
    n_0^2 = n_i^2 \frac{\mu_p}{\mu_n}
\end{equation*}
\begin{equation*}
    n_0 = n_i \sqrt{\frac{\mu_p}{\mu_n}}
\end{equation*}

\begin{equation*}
n_0 = n_i \sqrt{\frac{\mu_p}{\mu_n}}
\end{equation*}
\begin{equation*}
p_0 = n_i \sqrt{\frac{\mu_n}{\mu_p}}
\end{equation*}
We substitute these values into the expression for $ \sigma $:
\begin{equation*}
\sigma = q(n_0 \mu_n + p_0 \mu_p)
\end{equation*}
\begin{equation*}
\sigma = q(n_i \sqrt{\frac{\mu_p}{\mu_n}} \mu_n + n_i \sqrt{\frac{\mu_n}{\mu_p}} \mu_p)
\end{equation*}
\begin{equation*}
\sigma = q n_i (\sqrt{\mu_n \mu_p} + \sqrt{\mu_n \mu_p})
\end{equation*}
\begin{equation*}
\sigma_{\text{min}} = 2 q n_i \sqrt{\mu_n \mu_p}
\end{equation*}
Hence, the minimum conductivity $ \sigma_{\text{min}} $ is given by $ 2 q n_i \sqrt{\mu_n \mu_p} $ when $ n_0 = n_i \sqrt{\frac{\mu_p}{\mu_n}} $.

The intrinsic conductivity $\sigma_i$ for Si can be calculated using the equation:

\begin{equation*}
    \sigma_i = qn_i(\mu_n+\mu_p)
\end{equation*}

for silicon:

\begin{table}[H]
    \centering
    \begin{tabular}{cc}
    \hline
    Parameter & Value \\
    \hline
    \( \mu_n \) (Electron mobility in \(\text{cm}^2/\text{V-s}\)) & 1350 \\
    \( \mu_p \) (Hole mobility in \(\text{cm}^2/\text{V-s}\)) & 450 \\
    \( n_i \) (Intrinsic carrier concentration in \(\text{cm}^{-3}\)) & \(1.5 \times 10^{10}\) \\
    \( q \) (Elementary charge in C) & \(1.6 \times 10^{-19}\) \\
    \hline
    \( \sigma_i \) (Intrinsic conductivity in \(\text{S/cm}\)) & \(4.32 \times 10^{-2}\) \\
    $\sigma_{min}$(Minimum conductivity in \(\text{S/cm}\)) & $3.74 \times 10^{-2}$\\
    \hline
    \end{tabular}
    \caption{Typical values for silicon and calculated intrinsic conductivity}
\end{table}

when comparing the two conductivities we can se that extrinsic doesn't necessary mean that we get a more conductivity.  