\section{Problem 1 - Conductivity}

The conductivity of a semiconductor will depend on the doping concentration. Express the conductivity $\sigma$ as a function of

\begin{equation*}
     \tilde{n}=\frac{n_{0}}{n_{i}} 
\end{equation*}

i.e. the electron concentration normalized to the intrinsic value.

The general equation for conductivity due to electron is 

\begin{equation*}
    \sigma=q n_0 \mu_{n} 
\end{equation*}

from the given function we have that 

\begin{equation*}
    n_0=\tilde{n}\cdot n_i
\end{equation*}

This gives us

\begin{equation*}
    \sigma=q (\tilde{n}\cdot n_i) \mu_{n} 
\end{equation*}

\begin{equation*}
    \mu_{n} \equiv \frac{q \tau}{m_{n}^{*}}
\end{equation*}


\begin{equation*}
    \Rightarrow \sigma=(\tilde{n}\cdot n_i) \frac{q^2\tau}{m_{n}^{*}}
\end{equation*}

where $q$ is the charge of an electron, $\tilde{n}$ is the electron concentration normalized to the intrinsic value $n_i$ the intrinsic carrier concentration, $\mu_{n}$ is the electron mobility $\tau is $



