\renewcommand{\part}[1]{\textbf{\large Part \Alph{partCounter}}\stepcounter{partCounter}\\} % part macro
\newcommand{\solution}{\textbf{\large Solution}} % solution macro

% General writing
\newcommand{\bracket}[1]{\left(#1\right)} % for automatic resizing of brackets
\newcommand{\sbracket}[1]{\left[#1\right]} % for automatic resizing of square brackets
\newcommand{\mset}[1]{\left\{#1\right\}} % for automatic resizing of curly brackets
\newcommand{\defeq}{\coloneqq} % the "defined as" command
\newcommand{\ndash}{--\ } % for producing a dash
\newcommand{\mdash}{---\ } % for producing a longer dash
\newcommand{\by}{\times} % used for talking about m by n matrices
\newcommand{\eg}{e.g.\ } % for example
\newcommand{\ie}{i.e.\ } % i.e.
\newcommand{\RNum}[1]{\uppercase\expandafter{\romannumeral #1\relax}} % uppercase roman numerals
\renewcommand{\cite}{\textcite} % easier citing
\newcommand{\hide}[1]{} % useful for temporarily removing some code

\newcommand{\rednote}[1]{{\color{red} #1}} % for red text
\newcommand{\bluenote}[1]{{\color{blue} #1}} % for blue text
\newcommand{\greennote}[1]{{\color{green} #1}} % for green text

% Blackboard Maths Symbols
\newcommand{\N}{\mathbb{N}} % natural numbers
\newcommand{\Q}{\mathbb{Q}} % rational numbers
\newcommand{\Z}{\mathbb{Z}} % integers
\newcommand{\R}{\mathbb{R}} % real numbers
\newcommand{\C}{\mathbb{C}} % complex numbers
\newcommand{\E}{\mathbb{E}} % expectaion operators
\renewcommand{\P}{\mathbb{P}} % probability measures

% Calligraphic Maths Symbols
\newcommand{\F}{\mathcal{F}} % sigma-field

% Bold Maths Symbols
\newcommand{\1}{\mathbf{1}} % characteristic function

% Maths operations
\newcommand{\union}{\cup} % union
\newcommand{\intersect}{\cap} % intersection
\newcommand{\directsum}{\oplus} % direct sum
\newcommand{\Union}{\bigcup} % bigger union symbol
\newcommand{\Intersect}{\bigcap} % bigger intersection symbol
\newcommand{\Directsum}{\bigoplus} % bigger direct sum symbol
\newcommand{\tensor}{\otimes} % tensor
\newcommand{\Tensor}{\bigotimes} % bigger tensor symbol
\newcommand{\remove}{\setminus} % set difference
\renewcommand{\hom}[2]{\operatorname{Hom}\bracket{{#1}, {#2}}} % Hom sets

% Maths operations
\newcommand{\powerset}[1]{\mathcal{P}\bracket{#1}} % powerset
\renewcommand{\ip}[3]{\left \langle #1, #2 \right \rangle_{#3}} % inner product
\newcommand{\cardinality}[1]{\operatorname{Card}\bracket{#1}} % cardinality
\newcommand{\inv}[1]{#1^{-1}} % inverse
\newcommand{\adj}[1]{#1^{*}} % adjoint
\newcommand{\conj}[1]{\overline{#1}} % conjugate
\newcommand{\vspan}[1]{\operatorname{Span}\left\{#1\right\}} % vector space span
\renewcommand{\rank}[1]{\operatorname{Rank}\bracket{#1}} % rank of a linear transformation
\newcommand{\diag}[1]{\operatorname{Diag}\bracket{#1}} % diagonal elements of a square matrix
\renewcommand{\tr}[1]{\operatorname{Tr}\bracket{#1}} % trace
\newcommand{\kernel}[1]{\operatorname{Ker}\bracket{#1}} % kernel
\newcommand{\im}[1]{\operatorname{Im}\bracket{#1}} % image
\newcommand{\range}[1]{\operatorname{Ran}\bracket{#1}} % range
\newcommand{\col}[1]{\operatorname{Col}\bracket{#1}} % column space
\renewcommand{\exponential}[1]{\operatorname{exp}\bracket{#1}} % exponential function
\newcommand{\floor}[1]{\left\lfloor#1\right\rfloor} % macro for the floor function
\renewcommand{\ceil}[1]{\left\lceil#1\right\rceil} % macro for the ceiling function
\newcommand{\dx}{\,\mathrm{d}x} % differential of x
\newcommand{\dy}{\,\mathrm{d}y} % differential of y
\newcommand{\dt}{\,\mathrm{d}t} % differential of t


% optimisation objectives
\DeclareMathOperator*{\argmin}{argmin} % argument which minimises some criterion
\DeclareMathOperator*{\argmax}{argmax} % argument which maximises some criterion
\DeclareMathOperator*{\arginf}{arginf} % argument that gives the infimum of some criterion
\DeclareMathOperator*{\argsup}{argsup} % argument that gives the supremum of some criterion

% Statistical operations
\newcommand{\given}{\vert} % given (used in conditional probability)
\newcommand{\suchthat}{\;\middle|\;} % such that
\newcommand{\probspace}{\bracket{\Omega, \F, \P}} % a probability space
\newcommand{\borel}[1]{\B\bracket{#1}} % the Borel sigma-field
\newcommand{\prob}[1]{\P\bracket{#1}} % Probability
\newcommand{\cprob}[2]{\P\bracket{#1 \middle\given #2}} % Conditional probability
\newcommand{\expect}[1]{\E\bracket{#1}} % Expectation
\newcommand{\cexpect}[2]{\E\bracket{#1 \middle\given #2}} % Conditional expectation
\renewcommand{\var}[1]{\operatorname{Var}\bracket{#1}} % Variance
\newcommand{\cvar}[2]{\operatorname{Var}\bracket{#1 \middle\given #2}} % Conditional variance
\newcommand{\cov}[2]{\operatorname{Cov}\bracket{#1, #2}} % Covariance
\newcommand{\ccov}[3]{\operatorname{Cov}\bracket{#1, #2 \middle\given #3}} % Conditional covariance
\newcommand{\corr}[2]{\operatorname{Corr}\bracket{#1, #2}} % Correlation
\newcommand{\ccorr}[3]{\operatorname{Corr}\bracket{#1, #2 \middle\given #3}} % Conditional correlation

% Statistical symbols
\newcommand{\indep}{\perp\!\!\!\!\!\perp} % statistical independence
\newcommand{\sigfield}[1]{\sigma\bracket{#1}} % sigma-field
\newcommand{\evalue}[1]{\lambda_{#1}} % eigenvalue


% Images

\newcommand{\importimage}[1]{
    \begin{figure}[H]
        \centering
        \includegraphics[width=0.7 \textwidth]{../figures/#1}
        \label{fig:#1}
    \end{figure}}

\newcommand{\importimagewcaption}[2]{
    \begin{figure}[H]
        \centering
        \includegraphics[width=0.7 \textwidth]{../figures/#1}
        \caption*{#2}
        \label{fig:#1}
    \end{figure}}