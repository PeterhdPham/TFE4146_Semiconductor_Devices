\section{Problem 3 - The Haynes-Shockley Experiment}
\textbf{Figure \ref{fig:Haynes-Shockley_experiment.png} shows the experimental setup for the Haynes-Shockley experiment. The length of the sample is \( L_{0}=75 \mathrm{~mm} \) and the points (1) and (2) are separated by \( L=30 \mathrm{~mm} \). The supplied voltages are \( E_{0}=100 \mathrm{~V} \) and \( E_{2}=15 \mathrm{~V} \) and the resistors have the values \( R_{1}=10 \mathrm{k} \Omega \) and \( R_{2}=7 \mathrm{k} \Omega \) respectively.}

\importimagewcaption{Haynes-Shockley_experiment.png}{The setup for the Haynes-Shockley experiment}

\subsection*{a) Explain how we can use this experiment to determine the mobility and diffusion coefficient for holes in the n-type semiconductor.}

The basic principles of the Haynes-Shockley experiment are:

\begin{enumerate}
    \item A pulse of holes is created in a n-type bar that contain an electric field.
    \item As the pulse drifts in the field and spreads out by diffusion, the excess hole concentration is monitored at som point down the bar.
    \item The time required for the holes to drift a given distance in the field gives a measure of the mobility.
    \item The spreading of the pulse during a given time is used to calculate the diffusion coefficient.
\end{enumerate}

\importimagewcaption{Drift'ndiffusion.png}{Drift and diffusion of a hole pulse in an n-­type bar: (a) sample geometry; (b) position and shape of the pulse for several times during its drift down the bar.}

\subsection*{b) A very short pulse of holes is injected into the sample at point (1) at time \( t=0 \). Because of the applied voltage \( E_{0} \), the pulse of excess holes will be swept from (1) to (2). The voltage signal seen on the oscilloscope is shown in figure \ref{fig:voltage_signal.png} . The signal peak arrives after \( t_{d}=200 \mathrm{~ns} \).Calculate the mobility \( \mu_{p} \).}

\importimagewcaption{voltage_signal.png}{ The voltage signal seen on the oscilloscope}

The mobility $\mu$ is given by

\begin{equation*}
    \mu_p=\frac{V_d}{\mathscr{E}}
\end{equation*}

where $V_d$ is the drift velocity given by

\begin{equation*}
    V_d=\frac{L}{t_d}=1.5\cdot10^{5}\frac{m}{s}
\end{equation*}

the $\mathscr{E}$-field is given by

\begin{equation*}
    \mathscr{E}=\frac{E_0}{L_0}=\frac{100}{0.075}=1333.3\frac{V}{m}
\end{equation*}

This gives us

\begin{align*}
    \mu_p&=\frac{1.5\cdot10^{5}}{1333.3}\\
    \mu_p&=112.5 \frac{m^2}{V\cdot s}
\end{align*}

\subsection*{c) Assume \( t_{d}<<\tau_{p} \) such that recombination of charge carriers can be ignored. Show that the diffusion coefficient is given by}

\[D_{p}=\frac{(\Delta t L)^{2}}{16 t_{d}^{3}},\]
\textbf{and calculate \( D_{p} \) for this particular sample.Hint: Ignoring recombination, the diffusion equation, which describes the broadening of the pulse, reduces to}

\[\frac{\partial \delta p}{\partial t}=D_{p} \frac{\partial^{2} \delta p}{\partial x^{2}}\]

\textbf{which has the solution}
\[\delta p(x, t)=\frac{\Delta P}{2 \sqrt{\pi D_{p} t}} e^{\frac{-x^{2}}{4 D_{p} t}}\]

As the pulse drifts in the $\mathscr{E}$ field it aslo spreads out by diffusion. By examin the case of diffusion of a pulse without drift, neglecting recombination, the equation which the hole distribution must satisfy is the time-dependent diffusion equation

\begin{equation*}
    \frac{\partial \delta p}{\partial t}=D_{p} \frac{\partial^{2} \delta p}{\partial x^{2}}-\frac{\delta p}{\tau_{n}} 
\end{equation*}

For the case of negligible recombination, we can rewrite the diffusion equation as 


\begin{equation*}
    \frac{\partial \delta p}{\partial t}=D_{p} \frac{\partial^{2} \delta p}{\partial x^{2}}
\end{equation*}

The function which satisfies this qeuation is called a \textit{gaussioan distribution},

\begin{equation*}
    \delta p(x, t)=\left[\frac{\Delta P}{2 \sqrt{\pi D_{p} t}}\right] e^{-x^{2} / 4 D_{p} t}
\end{equation*}

where $\Delta P$ is the number of holes per unit area created over a negligibly small distance at $t=0$. The factor in brackets indicates that the peak value of the pulse at ($x=0$) decreases with time. We can redefine the peak value of the pulse as $\widehat{\delta p}$ at any given time, we can use the equationa bove too calculate $D_p$ at some point x. The most convenient choice is $\frac{\Delta x}{2}$ as this is where $\Delta p$ is down by $\frac{1}{e}$ of its peak value. There fore we can write

\begin{equation*}
    \begin{aligned} e^{-1} \widehat{\delta p} & =\widehat{\delta p} e^{-(\Delta x / 2)^{2} / 4 D_{p} t_{d}} \\ D_{p} & =\frac{(\Delta x)^{2}}{16 t_{d}}\end{aligned}
\end{equation*}
