\section{Problem 2 - Transient and A-C Conditions}
\textbf{Consider a $\mathrm{p}^{+}-\mathrm{n}$ junction, where $Q_p(t)$ denotes the time-dependent excess hole charge in the neutral n-region. }
\subsection*{a) Explain intuitively why $Q_p(t)$ and the diode current $i(t)$ is related by }
$$ i(t)=\frac{Q_p(t)}{\tau_p}+\frac{d Q_p(t)}{d t} $$ 

This relationship can be understood in two parts:

\begin{enumerate}
    \item \textbf{Steady-State Recombination Current:}
    \subitem The term $\frac{Q_p(t)}{\tau_p}$  represents the recombination current in the base region. Intuitively, this means that the charge carriers (electrons and holes) stored in the base region will recombine and annihilate each other at a rate that is inversely proportional to the lifetime $\tau_p$.The longer the lifetime, the slower the recombination. This term essentially says, "if we leave the stored charge $Q_p(t)$  in the base for a time $\tau_p$, all of it would recombine." So, the rate at which it's recombining at any instant is given by the fraction $\frac{Q_p(t)}{\tau_p}$.
    \item \textbf{Rate of Change of Stored Charge:}
    \subitem The term $\frac{d Q_p(t)}{d t}$ represents the rate of change of the stored charge in the base region. This is intuitive because any change in the stored charge $Q_p(t)$ will directly affect the current flowing through the diode. If the stored charge is increasing (i.e., $\frac{d Q_p(t)}{d t}>0$ ), this implies that more carriers are entering the base than leaving, which increases the diode current. Conversely, if the stored charge is decreasing (i.e., $\frac{d Q_p(t)}{d t}<0$ ), this means more carriers are leaving the base than entering, which reduces the diode current.
\end{enumerate}

To tie everything together:
\begin{itemize}
    \item The diode current is affected both by the charge currently stored in the base region and by how that charge is changing with time.
    \item The first term gives the contribution from the steady-state recombination of stored charge, while the second term gives the contribution from the dynamic change in stored charge.
\end{itemize}


\subsection*{b) Find the stored charge as a function of time when a forward current $I$ is suddenly switched on at time $t=0$. In other words, let }
$$ i(t)=\left\{
    \begin{array}{ll} 
        0 & \text { for } t<0 \\ 
        I & \text { for } t \geq 0 '
\end{array} .\right. $$

\textbf{Assume $Q_p(t)=0$ for all $t<0$ and find $Q_p(t)$ for $t \geq 0$. }

 We begin with Laplace transform

$$
\mathscr{L}\left\{i(t)\right\}=\mathscr{L}\left\{\frac{Q_p(t)}{\tau_p}\right\}+\mathscr{L}\left\{\frac{d Q_p(t)}{d t}\right\}
$$

For the term $\mathscr{L}\left\{\frac{Q_p(t)}{\tau_p}\right\}$:

$$
\mathscr{L}\left\{\frac{Q_p(t)}{\tau_p}\right\}=\frac{1}{\tau_p} \mathscr{L}\left\{Q_p(t)\right\}
$$

For the term $\mathscr{L}\left\{\frac{d Q_p(t)}{d t}\right\}$, using the property:

$$
\mathcal{L}\left\{\frac{d f(t)}{d t}\right\}=sF(s)-f(0)
$$

where $F(s)$ is the Laplace transform of $f(t)$ and $f(0)$ is the initial value of $f(t)$, we get:
$$
\mathcal{L}\left\{\frac{d Q_p(t)}{d t}\right\}=s \mathcal{L}\left\{Q_p(t)\right\}-Q_p(0)
$$

For the term $\mathscr{L}\left\{i(t)\right\}$:

$$
\mathscr{L}\left\{i(t)\right\}= \mathscr{L}\left\{I u(t)\right\}
$$
where $u(t)$ is the unit step function

$$
\mathscr{L}\left\{i(t)\right\}= \frac{I}{s}
$$


Combining the three results:
$$
\frac{I}{s}=\frac{1}{\tau_p} \mathcal{L}\left\{Q_p(t)\right\}+s \mathcal{L}\left\{Q_p(t)\right\}-Q_p(0)
$$

If we let $Q_p(s)$ be the Laplace transform of $Q_p(t)$, and apply the assumption of $Q_p(t)=0$ for $t\leq0$, then:
\begin{align*}
    \frac{I}{s}&=\left(\frac{1}{\tau_p} +s \right) Q_p(s)\\
    \frac{I}{s\left(\frac{1}{\tau_p} +s \right) }&=Q_p(s)\\
    \frac{I\tau_p}{s(s\tau_p)+1}&=Q_p(s)=\frac{A}{s}+\frac{B}{s\tau_p+1}
\end{align*}

this gives us 

$$
I\tau_p=A\left(s\tau_p+1\right)+Bs
$$
for $s=0$
$$I\tau_p=A$$

to find B:
\begin{align*}
    I\tau_p-A\left(s\tau_p+1\right)&=Bs\\
    -I\tau_p\left(s\tau_p\right)&=Bs\\
    B&=-I\tau_p^2
\end{align*}

$$Q_p(s)=\frac{I\tau_p}{s}-\frac{I\tau_p^2}{s\tau_p+1}$$

$$Q_p(t)=I\tau_p \left(1-e^{-\frac{t}{\tau_p}}\right)u(t)$$



\subsection*{c) At any time during the transient, the junction voltage $v(t)$ is related to $\Delta p_n(t)$ by }
$$ \Delta p_n(t)=p_n\left(e^{\frac{q v(t)}{k T}}-1\right) $$
\textbf{Use the quasi steady-state approximation 
$$ \delta p_n\left(x_n, t\right)=\Delta p_n(t) e^{-x_n / L_p} $$
 to find the junction voltage $v(t)$ as a function of time for the turn-on transient. Assume the neutral n-region is long compared to $L_p$. Sketch $v(t)$ and verify that your result is reasonable for $t \rightarrow \infty$.}

Using the quasi steady-state approximation
$$ \delta p_n\left(x_n, t\right)=\Delta p_n(t) e^{-x_n / L_p} $$
we have for the stored charge at any instant
$$
Q_p(t)=q A \int_0^{\infty} \Delta p_n(t) e^{-x_n / L_p} d x_n=q A L_p \Delta p_n(t)
$$

Relating $\Delta p_n(t)$ to $v(t)$ we have
\begin{align*}
    \Delta p_n(t)&=p_n\left(e^{q \nu(t) / k T}-1\right)=\frac{Q_p(t)}{q A L_p}\\
    e^{q \nu(t) / k T}&=\frac{Q_p(t)}{q A L_pp_n}+1\\
    v(t)&=\frac{kT}{q}\ln \left(\frac{I\tau_p \left(1-e^{-\frac{t}{\tau_p}}\right)u(t)}{q A L_pp_n}+1\right)
\end{align*}

Using arbitary values for the variables we get

\importimagewcaption{2c.png}{Plot of $v(t)$}
