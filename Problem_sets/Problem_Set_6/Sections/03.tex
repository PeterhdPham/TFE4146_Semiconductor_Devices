\section{Problem 3 - Narrow base diode}

\textbf{Assume that a $\mathrm{p}^{+}-\mathrm{n}$ diode is built with an $\mathrm{n}$-region width $l$ smaller than a hole diffusion length, i.e. $l<L_p$. This is the so-called narrow base diode. Since for this case holes are injected into a short n-region under forward bias, we cannot use the assumption $\delta p_n\left(x_n=\infty\right)=0$ as a boundary condition for the steady-state diffusion equation. Instead, assume that the minority carrier concentration is forced to zero at $x_n=l$ by an ohmic contact to the external circuit.}
\subsection*{a) Solve the diffusion equation for holes in the neutral n-region to obtain}
$$
\delta p_n\left(x_n\right)=\frac{\Delta p_n\left[e^{\left(l-x_n\right) / L_p}-e^{\left(x_n-l\right) / L_p}\right]}{e^{l / L_p}-e^{-l / L_p}}
$$

\textbf{Sketch $\delta p_n\left(x_n\right)$ for different values of $l / L_p$.}

In steady state cases the diffusion equations can be written as

$$
\frac{d^2 \delta p}{dx^2}\equiv\frac{\delta p}{L_p^2}
$$

Given the boundary conditions:
\begin{enumerate}
    \item At $x_n=0$:$\delta p_n(o)=\Delta p_n$
    \item At $x_n=l$:$\delta p_n(l)=0$
\end{enumerate}

The solution to this equation has the form 

$$\delta p(x)=C_1e^{\frac{x}{L_p}}+C_2e^{-\frac{x}{L_p}}$$

using the boundary conditions we get
\begin{align*}
    \Delta p_n&=C_1e^{\frac{0}{L_p}}+C_2e^{-\frac{0}{L_p}}\\
    C_2&=\Delta p_n-C_1
\end{align*}

\begin{align*}
    0&=C_1e^{\frac{l}{L_p}}+C_2e^{-\frac{l}{L_p}}\\
    C_2&=\frac{-C_1e^{\frac{l}{L_p}}}{e^{-\frac{l}{L_p}}}\\
    C_2&=-C_1e^{\frac{2l}{L_p}}
\end{align*}

\begin{align*}
    \Delta p_n-C_1&=-C_1e^{\frac{2l}{L_p}}\\
    \Delta p_n&=C_1-C_1e^{\frac{2l}{L_p}}\\
    \Delta p_n&=C_1\left(1-e^{\frac{2l}{L_p}}\right)\\
    C_1&=\frac{\Delta p_n}{1-e^{\frac{2l}{L_p}}}
\end{align*}

\begin{align*}
    C_2&=-\frac{\Delta p_n e^{\frac{2l}{L_p}}}{1-e^{\frac{2l}{L_p}}} 
\end{align*}

This gives us the solution

\begin{align*}
    \delta p_n(x_n)&=\frac{\Delta p_ne^{\frac{x_n}{L_p}}}{1-e^{\frac{2l}{L_p}}}-\frac{\Delta p_n e^{\frac{2l}{L_p}}e^{-\frac{x_n}{L_p}}}{1-e^{\frac{2l}{L_p}}}\\
    \delta p_n(x_n)&=\frac{\Delta p_n \left[e^{\frac{x_n}{L_p}}-e^{\frac{2l}{L_p}}e^{-\frac{x_n}{L_p}}\right] }{1-e^{\frac{2l}{L_p}}}\\
    \delta p_n(x_n)&=\frac{\Delta p_n \left[e^{\frac{x_n}{L_p}}e^{-\frac{l}{L_p}}-e^{\frac{2l}{L_p}}e^{-\frac{l}{L_p}}e^{-\frac{x_n}{L_p}}\right] }{e^{-\frac{l}{L_p}}-e^{\frac{2l}{L_p}}e^{-\frac{l}{L_p}}}\\
    \delta p_n(x_n)&=\frac{\Delta p_n \left[e^{\frac{x_n-l}{L_p}}-e^{\frac{l-x_n}{L_p}}\right] }{e^{-\frac{l}{L_p}}-e^{\frac{l}{L_p}}}\\
    \delta p_n(x_n)&=\frac{\Delta p_n \left[e^{\frac{l-x_n}{L_p}}-e^{\frac{x_n-l}{L_p}}\right] }{e^{\frac{l}{L_p}}-e^{-\frac{l}{L_p}}}\\
\end{align*}

\importimagewcaption{3a.png}{Plot of $\delta p_n\left(x_n\right)$ for different values of $l / L_p$.}


\subsection*{b) Show that the current in the diode is}
$$
I=\left(\frac{q A D_p p_n}{L_p} \operatorname{coth}\left(\frac{l}{L_p}\right)\right)\left(e^{q V / k T}-1\right) .
$$

\textbf{Compare the magnitude of the current of this narrow base diode to that of a long $\mathrm{p}^{+}-\mathrm{n}$ junction. \\Given:}
$$
\operatorname{coth}(x)=\frac{e^x+e^{-x}}{e^x-e^{-x}}
$$

The hole current density, $J_p$, is the sum of the drift and diffusion current densities, but in the neutral $n$-region, the drift current is zero because there is no electric field. Therefore, the hole current is given by the diffusion current:
$$
J_p=-q D_p \frac{d \delta p_n}{d x_n}
$$

Using the hole concentration, $\delta p_n\left(x_n\right)$, we derived in part a:

Differentiating $\delta p_n\left(x_n\right)$ with respect to $x_n$ :
$$
\frac{d \delta p_n}{d x_n}=\frac{\Delta p_n\left[-\frac{1}{L_p} e^{\frac{l-x_n}{L_p}}-\frac{1}{L_p} e^{\frac{x_n-l}{L_p}}\right]}{e^{\frac{l}{L_p}}-e^{-\frac{l}{L_p}}}
$$

Let's substitute this expression into the equation for $J_p$ to determine the hole current density:
$$
J_p=q D_p \frac{\Delta p_n\left[\frac{1}{L_p} e^{\frac{l-x_n}{L_p}}+\frac{1}{L_p} e^{\frac{x_n-l}{L_p}}\right]}{e^{\frac{1}{L_p}}-e^{-\frac{1}{L_p}}}
$$

At the $p^{+}-n$ junction, $x_n=0$. So:
$$
J_p(0)=q D_p \frac{\Delta p_n\left[\frac{1}{L_p} e^{\frac{1}{L_p}}+\frac{1}{L_p} e^{-\frac{1}{L_p}}\right]}{e^{\frac{1}{L_p}}-e^{-\frac{1}{L_p}}}
$$

Now, the total current $I$ is $J_p$ multiplied by the cross-sectional area, $A$
$$
I=A J_p(0)
$$

Substituting the given relation for $\operatorname{coth}(x)$ :
$$
\operatorname{coth}(x)=\frac{e^x+e^{-x}}{e^x-e^{-x}}
$$

The expression for the current $I$ in the diode is:
$$
I=\left(\frac{q A D_p p_n}{L_p} \cdot \frac{\left(e^{\frac{1}{L_p}}+e^{-\frac{1}{L_p}}\right)}{e^{\frac{1}{L_p}}-e^{-\frac{l}{L_p}}}\right)\left(e^{\frac{q V}{k T}}-1\right)
$$

Using the given relation for $\operatorname{coth}(x)$ :
$$
\operatorname{coth}(x)=\frac{e^x+e^{-x}}{e^x-e^{-x}}
$$

The above equation can be simplified as:
$$
I=\left(\frac{q A D_p p_n}{L_p} \cdot \operatorname{coth}\left(\frac{l}{L_p}\right)\right)\left(e^{\frac{q V}{k T}}-1\right)
$$

Which matches the desired result.

Now, let's compare the magnitude of the current of this narrow base diode to that of a long $p^{+}-n$ junction.

For a long $p^{+}-n$ junction, the current is given by:
$$
I=\frac{q A D_p p_n}{L_p}\left(e^{\frac{q V}{k T}}-1\right)
$$

Comparing the two expressions for $I$, we can see that the current in the narrow base diode is modified by a factor of:
$$
\operatorname{coth}\left(\frac{l}{L_p}\right)
$$

The hyperbolic cotangent function, $\operatorname{coth}(x)$, approaches 1 as $x$ approaches 0 , and as $x$ becomes large, $\operatorname{coth}(x)$ approaches 1 . This means that for a very short $n$-region ( $l$ is small compared to $L_p$ ), the current in the narrow base diode will be close to that of a long $p^{+}-n$ junction. As $l$ increases, the factor $\operatorname{coth}\left(\frac{l}{L_p}\right)$ will continue to be close to 1 , meaning the current will remain close to that of the long junction.

In conclusion, the current in the narrow base diode is modified by the factor $\operatorname{coth}\left(\frac{l}{L_p}\right)$, but for many practical cases, it will be close to the current in a long $p^{+}-n$ junction.