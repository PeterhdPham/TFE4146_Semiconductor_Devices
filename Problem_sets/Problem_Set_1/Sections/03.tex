\section{Part 3: Carrier Concentrations}

\subsection*{Problem 4}

\subsubsection*{a) State the fermi-dirac distribution and explain what it describes.}

The fermi-dirac distribution describes the probability for if a given energy state is occupied by a electron at a given temperature. It is given by the equation:

\begin{equation*}
    f(E)=\frac{1}{e^{\frac{E-E_F}{k_B T}}+1}
\end{equation*}

Where $f(E)$ is the probability for if a given energy state with the energy $E$ is occupied by a electron. $E_F$ is the Fermi-energy, $k_B$ is Boltzmanns constant, and $T$ is the temperature.

\subsection*{b) The general expression for calculating the concentration of electrons in the conduction band is}

\begin{equation*}
    n_{0}=\int_{E_{c}}^{\infty} f(E) N(E) d E \
\end{equation*}

\textbf{Explain the terms and parameters in the equation above and use it to derive the simplified expression}

\begin{equation*}
    n_{0}=N_{c} e^{-\left(E_{c}-E_{F}\right) / k_{B} T}
\end{equation*}

In this equation, the following terms are present:

\begin{enumerate}
    \item $n_0$ is the total number of electrons in the conduction band per unit volume.
    \item $E_c$ is the minimum energy required for an electron to be in the conduction band.
    \item $\int_{E_c}^{\infty}$ is an integral over all the energy states starting from $E_c$ up to infinity. This integral sums up the probability of occupation times the number of available states at that energy.
    \item $f(E)$ is the Fermi-Dirac distribution function, which gives the probability of occupation of an energy state $E$ at a given temperature. 
    \item $N(E)$ is the density of available states at a given energy $E$
\end{enumerate}

In this case we have that

\begin{equation*}
    N_{C}(E)=4 \pi\left(\frac{2 m^{*}}{h^{2}}\right)^{\frac{3}{2}} \cdot E^{1 / 2}
\end{equation*}

and

\begin{equation*}
    f(E)=\frac{1}{1+e^{\left(E-E_{F}\right) / k_{B} T}}
\end{equation*}

We aim to derive the simplified expression for $ n_0 $:

\begin{equation*}
n_{0} = N_c e^{-(E_{c} - E_{F}) / k_B T}
\end{equation*}

To proceed, we insert the expressions for $ f(E) $ and $ N(E) $ into the original equation and simplify:

\begin{align*}
n_0 &= \int_{E_c}^{\infty} \frac{N_c}{1 + e^{(E - E_F) / (k_B T)}} \, dE \\
&= N_c \int_{E_c}^{\infty} \frac{1}{1 + e^{(E - E_F) / (k_B T)}} \, dE
\end{align*}

We perform the integral by making the substitution $ u = (E - E_F) / (k_B T) $ which implies $ dE = k_B T \, du $:

\begin{align*}
n_0 &= N_c k_B T \int_{(E_c - E_F)/(k_B T)}^{\infty} \frac{1}{1 + e^u} \, du \\
&\approx N_c k_B T \int_{-\infty}^{0} e^u \, du \\
&= N_c k_B T [-e^u]_{-\infty}^0 \\
&= N_c (1) \\
&= N_c e^{-(E_c - E_F) / (k_B T)}
\end{align*}

Note: The approximation step where the integral limits change assumes that $ E_c $ is sufficiently greater than $ E_F $ such that the Fermi-Dirac distribution is almost zero beyond $ E_c $.

Thus, we have derived the simplified expression:

\begin{equation*}
n_0 = N_c e^{-(E_c - E_F) / (k_B T)}
\end{equation*}

\subsection*{c) Show that the equation above may be written as}

\begin{equation*}
    n_{0}=n_{i} e^{\left(E_{F}-E_{i}\right) / k_{B} T} 
\end{equation*}

\textbf{where $n_i$ and $E_i$ are the intrinsic electron concentration and intrinsic fermi-level, respectively.}

First, let's define the intrinsic electron concentration $ n_i $ in terms of $ N_c $ and $ E_c $:

\begin{equation*}
n_{i} = N_{c} e^{-\left(E_{c} - E_{i}\right) / k_{B} T}
\end{equation*}

Now, substitute this definition into the original equation for $ n_0 $:

\begin{equation*}
n_{0} = N_{c} e^{-\left(E_{c} - E_{F}\right) / k_{B} T}
\end{equation*}

Divide both sides by $ n_i $:

\begin{equation*}
\frac{n_{0}}{n_{i}} = \frac{N_{c} e^{-\left(E_{c} - E_{F}\right) / k_{B} T}}{N_{c} e^{-\left(E_{c} - E_{i}\right) / k_{B} T}}
\end{equation*}

Simplifying, we get:

\begin{equation*}
\frac{n_{0}}{n_{i}} = e^{\left(E_{i} - E_{F}\right) / k_{B} T}
\end{equation*}

Multiply both sides by $ n_i $:

\begin{equation*}
n_{0} = n_{i} e^{\left(E_{F} - E_{i}\right) / k_{B} T}
\end{equation*}

Thus, we have shown that the original equation can be rewritten in the desired form.


\subsection*{d) Show that the intrinsic fermi level $E_i$ is in the middle of the bandgap if and only if the effective densities of states $N_c$ and $N_v$ are equal}

We know the expressions for the intrinsic carrier concentrations for electrons \( n_i \) and holes \( p_i \) as:

\begin{equation*}
n_i = N_c e^{-\left(E_c - E_i\right) / k_B T}
\end{equation*}

\begin{equation*}
p_i = N_v e^{-\left(E_i - E_v\right) / k_B T}
\end{equation*}

For intrinsic semiconductors, \( n_i = p_i \). Setting these equal gives:

\begin{equation*}
N_c e^{-\left(E_c - E_i\right) / k_B T} = N_v e^{-\left(E_i - E_v\right) / k_B T}
\end{equation*}

Taking the logarithm of both sides yields:

\begin{equation*}
-\left(E_c - E_i\right) / k_B T + \ln N_c = -\left(E_i - E_v\right) / k_B T + \ln N_v
\end{equation*}

Simplifying, we find:

\begin{equation*}
(E_c - E_i) - (E_i - E_v) = k_B T (\ln N_c - \ln N_v)
\end{equation*}

Further simplifying:

\begin{equation*}
E_c - E_v = k_B T \ln\left(\frac{N_c}{N_v}\right)
\end{equation*}

For \( E_i \) to be in the middle of the bandgap, \( E_c - E_i = E_i - E_v \) or \( E_i = (E_c + E_v)/2 \).

This condition is met if \( \ln(N_c/N_v) = 0 \) or \( N_c = N_v \).

Thus, \( E_i \) is in the middle of the bandgap if and only if \( N_c = N_v \).

\subsection*{e) Use}

\begin{equation*}
    \begin{array}{l}N_{C}=2\left(\frac{2 \pi m_{n}^{*} k_{B} T}{h^{2}}\right)^{\frac{3}{2}} \\ N_{V}=2\left(\frac{2 \pi m_{p}^{*} k_{B} T}{h^{2}}\right)^{\frac{3}{2}}\end{array} 
\end{equation*}

\textbf{to show that the intrinsic Fermi level $E_i$ is positioned below the middle of the bandgap by}

\begin{equation*}
    \Delta E=k_{B} T \ln \left(\frac{m_{n}^{*}}{m_{p}^{*}}\right)^{\frac{3}{4}}
\end{equation*}

We previously derived that:

\begin{equation*}
E_c - E_v = k_B T \ln\left(\frac{N_c}{N_v}\right)
\end{equation*}

Substitute the expressions for \( N_C \) and \( N_V \):

\begin{equation*}
E_c - E_v = k_B T \ln\left(\frac{2\left(\frac{2 \pi m_{n}^{*} k_B T}{h^2}\right)^{\frac{3}{2}}}{2\left(\frac{2 \pi m_{p}^{*} k_B T}{h^2}\right)^{\frac{3}{2}}}\right)
\end{equation*}

Simplifying, we get:

\begin{equation*}
E_c - E_v = k_B T \ln\left(\left(\frac{m_{n}^{*}}{m_{p}^{*}}\right)^{\frac{3}{2}}\right)
\end{equation*}

\begin{equation*}
E_c - E_v = \frac{3}{2} k_B T \ln\left(\frac{m_{n}^{*}}{m_{p}^{*}}\right)
\end{equation*}

For \( E_i \) to be at the middle of the bandgap, we would have:

\begin{equation*}
E_i = \frac{E_c + E_v}{2}
\end{equation*}

Due to the effective mass difference, \( E_i \) will actually be shifted by \( \Delta E \) below the middle:

\begin{equation*}
E_i = \frac{E_c + E_v}{2} - \Delta E
\end{equation*}

Where \( \Delta E \) is:

\begin{equation*}
\Delta E = \frac{1}{2} \times \frac{3}{2} k_B T \ln\left(\frac{m_{n}^{*}}{m_{p}^{*}}\right)
\end{equation*}

\begin{equation*}
\Delta E = k_B T \ln\left(\frac{m_{n}^{*}}{m_{p}^{*}}\right)^{\frac{3}{4}}
\end{equation*}

Thus, we have shown that \( \Delta E \) is the amount by which the intrinsic Fermi level is positioned below the middle of the bandgap.

\subsection*{f) Calculate this deviation (from the midgap energy) for Ge and GaAs.}

Effective masses for Ge: $m_n^* = 0.56m_0;m_p^*=0.29m_0$\\
Effective masses for GaAs: $m_n^* = 0.067m_0;m_p^*=0.48m_0$

To calculate the deviation \( \Delta E \) for Ge and GaAs, we use the formula:

\begin{equation*}
\Delta E = k_B T \ln\left(\frac{m_{n}^{*}}{m_{p}^{*}}\right)^{\frac{3}{4}}
\end{equation*}

Here \( k_B = 8.617 \times 10^{-5} \, \text{eV/K} \) is the Boltzmann constant, and we assume room temperature \( T = 300 \, \text{K} \).

\textbf{For Ge:}

Effective masses for Ge are \( m_n^* = 0.56m_0 \) and \( m_p^* = 0.29m_0 \).

\begin{equation*}
\Delta E_{\text{Ge}} = 8.617 \times 10^{-5} \times 300 \ln\left(\frac{0.56}{0.29}\right)^{\frac{3}{4}}
\end{equation*}

After calculations, we find \( \Delta E_{\text{Ge}} \approx 0.010 \, \text{eV} \).

\textbf{For GaAs:}

Effective masses for GaAs are \( m_n^* = 0.067m_0 \) and \( m_p^* = 0.48m_0 \).

\begin{equation*}
\Delta E_{\text{GaAs}} = 8.617 \times 10^{-5} \times 300 \ln\left(\frac{0.067}{0.48}\right)^{\frac{3}{4}}
\end{equation*}

After calculations, we find \( \Delta E_{\text{GaAs}} \approx -0.029 \, \text{eV} \).

Thus, we have calculated the deviations \( \Delta E \) from the midgap energy for Ge and GaAs.
