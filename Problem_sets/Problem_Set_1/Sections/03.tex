\section{Part 3: Carrier Concentrations}

\subsection*{Problem 4}

\subsubsection*{a) State the fermi-dirac distribution and explain what it describes.}

The fermi-dirac distribution describes the probability for if a given energy state is occupied by a electron at a given temperature. It is given by the equation:

\begin{equation*}
    f(E)=\frac{1}{e^{\frac{E-E_F}{k_B T}}+1}
\end{equation*}

Where $f(E)$ is the probability for if a given energy state with the energy $E$ is occupied by a electron. $E_F$ is the Fermi-energy, $k_B$ is Boltzmanns constant, and $T$ is the temperature.

\subsection*{b) The general expression for calculating the concentration of electrons in the conduction band is}

\begin{equation*}
    n_{0}=\int_{E_{c}}^{\infty} f(E) N(E) d E \
\end{equation*}

\textbf{Explain the terms and parameters in the equation above and use it to derive the simplified expression}

